
\documentclass{sig-alternate}

\usepackage[none]{hyphenat}
\sloppy

\begin{document}
\conferenceinfo{\textit{MediaEval 2015 Workshop,}}{Sept. 14-15, 2015, Wurzen, Germany}

\title{DCLab at MediaEval2015 Retrieving Diverse \\ Social Images Task}

\numberofauthors{3}

\author{
\alignauthor
Zsombor Par\'oczi\\
       \affaddr{Budapest University of Technology and Economics}\\
       \email{paroczi@tmit.bme.hu}
\alignauthor
Kis-Kir\'aly M\'at\'e \\
		\affaddr{Budapest University of Technology and Economics}\\
		\email{kis.kiraly.mate@gmail.com>}
\alignauthor
D\'aniel Mira\\
		\affaddr{Budapest University of Technology and Economics}\\
		\email{miradaniellevente@gmail.com}
}

\maketitle
\begin{abstract}
TODO abstract
\end{abstract}

\section{Introduction}

When a potential tourist makes a search on a place, he (or she) want's to have a diverse and relevant visual result as a summary of the different views of the location. 

In the official challenge (Retrieving Diverse Social Images at MediaEval 2015: Challenge, Dataset and Evaluation) \cite{Task2015} a ranked list of location photos retrieved from Flickr is given, and the task is, to refine the results by providing a set of images that are both relevant and provide a diversified summary. The diversity means that images can illustrate different views of the location at different times of the day/year and under different weather conditions, creative views, etc. The goodness of the refinement process can be measured using the precision and diversity metric~\cite{Taneva:2010:GRP:1718487.1718541}. 

Our team participated in previous challanges \cite{szHucs2013bmemtm,Paroczi2014}, each year we experimented with a different approach. In 2013 we used diversification of initial results using clustering, but our solution was focused on diversification only. In 2014 we tried to focused on relevance and diversity with the same importance, as a new idea.

In this paper and in this year's challange we used a more sofisticated distance based clustering method, where we crafted the distance matrix ourself rather then using some n dimension based clustering with an eucledian distance function.

\section{Runs}

\subsection{Run1: Visual based reranking}

\subsection{Run2: Text based reranking}

\subsection{Run3: Text + Visual}

\subsection{Run4: Credibility based reranking}

\section{Results and Conclusion}

\begin{table}[h]
	\centering
\begin{tabular}{|l|r|r|r|}
	\hline 
	run name & P@20 & CR@20 & F1@20\tabularnewline
	\hline 
	\hline 
	Visual single & .7022 & .3702 & .4751\tabularnewline
	\hline 
	Visual multi & .7164 & .3857 & .4813\tabularnewline
	\hline 
	Text single & .6435 & .3494 & .4379\tabularnewline
	\hline 
	Text multi & .7021 & .3813 & .4748\tabularnewline
	\hline 
	Vistext single & .6732 & .3563 & .4554\tabularnewline
	\hline 
	Vistext multi & .6993 & .3683 & .4651\tabularnewline
	\hline 
	Cred single & .7014 & .3589 & .4651\tabularnewline
	\hline 
	Cred multi & .7150 & .3498 & .4479\tabularnewline
	\hline 
\end{tabular}
\label{table:results}
\caption{Average results of the approaches}
\end{table}

\begin{figure}[h]
\includegraphics[width=1.0\linewidth]{f1}
\caption{F1@N results}
\label{fig:f1}
\end{figure}

Our results can be seen on Table \ref{table:results}. and the F1 metrics can be seend on Figure \ref{fig:f1}, we listed the single and multi concept based separately. 

As you can see the visual based results are the best among the results, from the devset we found out, that the text based information for the images are in a lot of cases missing, or don't represent the images well. It's not uncommon that an author gives the same textual information to all of the images in a topic.

The credibility based results are better, then we expected, since our method basicly eliminates some authors from the results, which can have a negative effect on the CR metric.

\bibliographystyle{abbrv}
\bibliography{sigproc}

\end{document}
