
\documentclass{sig-alternate}

\usepackage[none]{hyphenat}
\sloppy

\begin{document}
\conferenceinfo{\textit{MediaEval 2015 Workshop,}}{Sept. 14-15, 2015, Wurzen, Germany}

\title{DCLab at MediaEval2015 Retrieving Diverse \\ Social Images Task}

\numberofauthors{3}

\author{
\alignauthor
Zsombor Par\'oczi\\
       \affaddr{Budapest University of Technology and Economics}\\
       \email{paroczi@tmit.bme.hu}
\alignauthor
Kis-Kir\'aly M\'at\'e \\
		\affaddr{Budapest University of Technology and Economics}\\
		\email{kis.kiraly.mate@gmail.com>}
\alignauthor
D\'aniel Mira\\
		\affaddr{Budapest University of Technology and Economics}\\
		\email{miradaniellevente@gmail.com}
}

\maketitle
\begin{abstract}
TODO abstract
\end{abstract}

\section{Introduction}

Many potential tourists do websearches when they try to find more information about a place he (or she) is potentially visiting. These people have only a vague idea about the location, knowing the name of the place. Our aim is to help them by providing a set of photos, as summary of the different views of the location. 

In the official challenge (Retrieving Diverse Social Images at MediaEval 2014: Challenge, Dataset and Evaluation) \cite{ionescu2014retrieving} a ranked list of location photos retrieved from Flickr (using text information) is given, and the task is, to refine the results by providing a set of images that are both relevant and provide a diversified summary. The diversity means that images can illustrate different views of the location at different times of the day/year and under different weather conditions, creative views, etc. ...

The goodness of the refinement process can be measured using the precision and diversity metric~\cite{Taneva:2010:GRP:1718487.1718541}. Earlier we have solved a very similar problem by diversification of initial results using clustering~\cite{szHucs2013bmemtm}, but our solution was focused on diversification only. The largest development of this paper is that we focus on both the relevance and diversity.


When processing an ordering (from the test data set) we give the relevance score of $p_k$ to the $k$th element of the ordering.

\section{Results and Conclusion}

\begin{table}[h]
\begin{tabular}{|l|r|r|r|}
	\hline 
	run name & P@20 & CR@20 & F1@20\tabularnewline
	\hline 
	\hline 
	VisClusterAvgRelevance & .7602 & .4107 & .5259\tabularnewline
	\hline 
	TextClusterAvgRelevance & .7809 & .4065 & .527\tabularnewline
	\hline 
	VisTextClusterAvgRelevance & .7756 & .4127 & .5305\tabularnewline
	\hline 
	VisTextClusterCredRelevance & .7415 & .3651 & .4819\tabularnewline
	\hline 
	VisTextClusterMixedRelevance & .7431 & .3682 & .4866\tabularnewline
	\hline 
\end{tabular}
\caption{Average results of the five approaches}
\end{table}

\includegraphics[width=0.9\linewidth]{p}

\includegraphics[width=0.9\linewidth]{cr}

\includegraphics[width=0.9\linewidth]{f1}


\bibliographystyle{abbrv}
\bibliography{sigproc}

\end{document}
