
\documentclass{acm_proc_article-me}

\usepackage[none]{hyphenat}
\sloppy

\begin{document}
\conferenceinfo{\textit{MediaEval 2015 Workshop,}}{Sept. 14-15, 2015, Wurzen, Germany}

\title{DCLab at MediaEval 2015 Retrieving Diverse \\ Social Images Task}

\numberofauthors{3}

\author{
\alignauthor
Zsombor Par\'oczi\\
       \affaddr{Budapest University of Technology and Economics}\\
       \email{paroczi@tmit.bme.hu}
\alignauthor
M\'at\'e Kis-Kir\'aly \\
		\affaddr{Budapest University of Technology and Economics}\\
		\email{kis.kiraly.mate@gmail.com}
\alignauthor
B\'alint Fodor\\
		\affaddr{Budapest University of Technology and Economics}\\
		\email{balint.fodor@gmail.com}
}

\maketitle
\begin{abstract}
In this paper we present our contribution to the MediaEval 2015 Retrieving Diverse Social Images Task which requested participants to provide methods for refining Flickr image retrieval results thus to increase their relevance and diversification. Our approach is based on re-ranking the original result, using a precomputed distance matrix and a spectral clustering scheme. We use color related visual features, text and credibility descriptors to define similarity between images.
\end{abstract}

\section{Introduction}

When a potential tourist makes an image search for a place, she expects to get a diverse and relevant visual result as a summary of the different views of the location. 

In the official challenge (Retrieving Diverse Social Images at MediaEval 2015)~\cite{Task2015} a ranked list of location photos retrieved from Flickr is given, and the task is to refine the result by providing a set of images that are both relevant and provide a diversified summary. An extended explanation for the task objectives, provided dataset and evaluation descriptors can be found in the task description paper \cite{Task2015}. The diversity means that images can illustrate different views of the location at different times of the day/year and under different weather conditions, creative views, etc. The utility score of the refinement process can be measured using the precision and diversity metric~\cite{Taneva2010}. 

Our team participated in previous challenges~\cite{szHucs2013bmemtm,Paroczi2014}, each year we experimented with a different approach. In 2013 we used diversification of initial results using clustering, but our solution was focused on diversification only~\cite{szHucs2013bmemtm}. In 2014 we tried to focus on relevance and diversity with the same importance as a new idea~\cite{Paroczi2014}.

In the previous approaches to the task we treated our feature vectors (calculated values from metrics) as an $N$ dimensional continuous space with Euclidean coordinates. In this year apporach we will define a set of hand crafted distance matrices with non-Euclidean coordinates, which can be used during the clustering. 

\section{Runs}

\subsection{Run1: Visual based re-ranking}\label{run1}
In the first run participants could use only visual based descriptors or own descriptors calculated using only the images.

Our main approach was using color based distances~\cite{Datta2008,Paramita2010} and filtering photos with faces~\cite{szHucs2013bmemtm,Paroczi2014}. We used two of the descriptors provided by the organizers~\cite{Task2015}:
\begin{description}
	\item[CM] Global Color Moments on HSV Color Space: represent the first three central moments of an image color distribution: mean, standard deviation and skewness
	\item [CN] Global Color Naming Histogram: maps colors to 11 universal color names: "black", "blue", "brown", "grey", "green", "orange", "pink", "purple", "red", "white", and "yellow"
\end{description}

We experimented with HOG feature distances but did not achieve any additional improvement.

First we calculated a new descriptor for each image: the $FACE$ descriptor is the ratio of the calculated area occupied by the possible face regions on an image and whole
image area~\cite{szHucs2013bmemtm}. Then we used the $CN$ descriptor to filter out black color based images, since mostly dark images tend to have less colors and those are mainly shifted into the gray range rather than having bright colors.

In the reordering step we started from the original result. We did our initial filtering by putting images to the end of the result list where $FACE>0$ or $CN[0]>0.8$, the first value in $CN$ corresponds to the color black.

After the preprocessing step we built the distance matrix $F$, between each $A$ and $B$ images the distance was calculated using the following equation:

\begin{gather*} 
F_{A,B} = \sum_{i=0}^{10}  \left | CN_{A}[i]-CN_{B}[i] \right | +\sum_{i=0}^{10} \left | s_i * (CM_{A}[i]-CM_{B}[i]) \right |\\
s_i = \left\{\begin{matrix}
5 \textup{, where }0 \leqslant  i < 3\\ 
1.5 \textup{, where }3 \leqslant  i < 5\\ 
0.5 \textup{, where }5 \leqslant  i < 9
\end{matrix}\right.
\end{gather*}

After the distance matrix was created we used unsupervised spectral clustering~\cite{Ma2009,Ng01onspectral} to create clusters from the first 150 images, the target cluster count was 10. 

The final result was generated by picking the lowest ranking item from each cluster, appending those to the result list, then repeating this until all the items are used. The same clustering and sorting method was used during run2 and run3.

\subsection{Run2: Text based re-ranking}\label{run2}
The second run was the text based re-ranking which is accomplished using the title, tags and description fields of each image.

As a preprocessing step we executed a stop word filtering. We also removed some special characters (namely: .,-:;0123456789()\_@) and HTML specific character sets (\&amp;, \&quot; and everything between < and >), then we used the remaining text as the input for a simple TF-IDF calculation~\cite{Yeh2008}. 

 We calculated the distance between images (e.g. description fields) $A$ and $B$ in the following manner. We initalize distance $G_{A,B}$ to zero and compared $A$ and $B$ at the term level. All occurring $t$ terms in document $A$ compared with all terms in the document $B$ and so on. If term $t$ is contained by both documents, then $G_{A,B}$ will not be increased. If $t$ contained by only one document, we take into consideration the document frequency ($DF_t$): if $DF_t<5$, then it is a rare term and $G_{A,B}$ should be increased by $2$; if $DF_t>DN/4$, then it is a common term and  $G_{A,B}$ should be increased by $0.1$ (where $DN$ is the total number of documents). If the term is not common nor rare, then we added the $DF_t/DN$ to the distance. 
 
Using the three text descriptors we created a weighted sum for the field distences, where the weights are as follows: title=$1$, tags=$2$, description=$0.5$. From these $G_{A,B}$ values we created the $G$ distance matrix.

\subsection{Run3: Multimodal re-ranking}
In the third run both visual and textual descriptors could be used to create the results.

We used our visual distance matrix $F$ (see Section \ref{run1}) and text distance matrix $G$ (see Section \ref{run2}) and created a new aggregate matrix $H$. This matrix is simply the sum of the corresponding values from both $F$ and $G$ matrix. We tried different kind of weighting methods, but the pure matrices supplied the best results on the devset.

\subsection{Run4: Credibility based re-ranking}
In the fourth run participants were provided with credibility descriptors \cite{Task2015}.

Using the original result we filtered the images by users who had $faceProportion$ more than $1.3$ to create the same effect as we did with the $FACE$ descriptor. With the purpose of increasing the diversity we used the $locationSimilarity$ descriptor, if this value exceeds the threshold of $3.0$ we excluded the image. Despite our simple approach we had great results on the devset.

\section{Results and Conclusion}

\begin{table}[t]
	\centering
\begin{tabular}{|l|r|r|r|}
	\hline 
	run name & P@20 & CR@20 & F1@20\tabularnewline
	\hline 
	\hline 
	Run1 single & .7022 & .3702 & .4751\tabularnewline
	\hline 
	Run1 multi & .7164 & .3857 & .4813\tabularnewline
	\hline 
	Run2 single & .6435 & .3494 & .4379\tabularnewline
	\hline 
	Run2 multi & .7021 & .3813 & .4748\tabularnewline
	\hline 
	Run3 single & .6732 & .3563 & .4554\tabularnewline
	\hline 
	Run3 multi & .6993 & .3683 & .4651\tabularnewline
	\hline 
	Run4 single & .7014 & .3589 & .4651\tabularnewline
	\hline 
	Run4 multi & .7150 & .3498 & .4479\tabularnewline
	\hline
\end{tabular}
\label{table:results}
\caption{Official results on the test data.}
\end{table}

\begin{figure}[t]
\includegraphics[width=1.0\linewidth]{f1}
\caption{Official runs F1-score metric for various cutoff points (results of test data).}
\label{fig:f1}
\end{figure}

Our results can be seen in Table \ref{table:results}. and the F1 metrics can be seen in Figure \ref{fig:f1}, we listed the single and multi-concept based results separately. 

As one can see the visual information based results are the best among all the runs. In the devset we experienced that the textual information for many images are missing or do not describe the content very well. It is not uncommon that an author gives the same textual information to all of the images in a topic.

\bibliographystyle{abbrv}
\bibliography{sigproc}

\end{document}
